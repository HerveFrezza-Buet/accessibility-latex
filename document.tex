\newif\ifzoom
\zoomfalse

%%%%%%%%%%%%%%%%%%%%%%%%%%%%%%%%%%
%                                %
% Select accessibility mode here %
%                                %
%%%%%%%%%%%%%%%%%%%%%%%%%%%%%%%%%%

%% Uncomment the following line to generate the accessible version.
%% \zoomtrue


%%%%%%%%%%%%%%%%%%%%%%%%%%%%%%%
%                             %
% Accessibility related stuff %
%                             %
%%%%%%%%%%%%%%%%%%%%%%%%%%%%%%%

\ifzoom
\documentclass[a4paper,17pt]{extbook} % or extarticle, extreport....
% sudo apt install texlive-fonts-recommended
\usepackage[scaled]{helvet}
\renewcommand\familydefault{\sfdefault} 
\usepackage[T1]{fontenc}
\newcommand{\IncludeGraphics}[2]{\includegraphics[width=\textwidth,height=150mm,keepaspectratio]{#2}}
\usepackage[margin=10mm, top=20mm]{geometry}
\else
\documentclass[a4paper,10pt,twoside]{book} % or article, report....
\newcommand{\IncludeGraphics}[2]{\includegraphics[#1]{#2}}
\usepackage[margin=3cm]{geometry}
\fi

%%%%%%%%%%%%%%%%%%%%%%%%%%%%%%
%                            %
% Usual latex file now stuff %
%                            %
%%%%%%%%%%%%%%%%%%%%%%%%%%%%%%


\usepackage[utf8]{inputenc}
\usepackage{graphicx}

\begin{document}

%% This will make math stuff bold.
\ifzoom
\mathversion{bold} 
\else
\fi

%% Title page 

\thispagestyle{empty}
\vfill
\hrule
\vspace{20mm}
\centerline{\Huge \sc \LaTeX\ accessibility example}
\vspace{20mm}
\hrule
\vspace{10mm}
\centerline{\hfill Last revision: \today}
\vfill
\strut
\pagebreak


%% Content

\chapter{Text and figures}

%% dummy text... Edgar Poe indeed.

At Paris, just after dark one gusty evening in the autumn of 18-, I was
enjoying the twofold luxury of meditation and a meerschaum, in company
with my friend C. Auguste Dupin, in his little back library, or
book-closet, au troisiême, No. 33, Rue Dunôt, Faubourg St. Germain. For
one hour at least we had maintained a profound silence; while each, to
any casual observer, might have seemed intently and exclusively occupied
with the curling eddies of smoke that oppressed the atmosphere of the
chamber. For myself, however, I was mentally discussing certain topics
which had formed matter for conversation between us at an earlier period
of the evening; I mean the affair of the Rue Morgue, and the mystery
attending the murder of Marie Rogêt. I looked upon it, therefore, as
something of a coincidence, when the door of our apartment was thrown
open and admitted our old acquaintance, Monsieur G--, the Prefect of the
Parisian police.

\begin{equation}
\sum_{i=1}^n \xi^2 \neq \int_0^1 x^{\xi+1}
\end{equation}

We gave him a hearty welcome; for there was nearly half as much of the
entertaining as of the contemptible about the man, and we had not seen
him for several years. We had been sitting in the dark, and Dupin now
arose for the purpose of lighting a lamp, but sat down again, without
doing so, upon G.'s saying that he had called to consult us, or rather
to ask the opinion of my friend, about some official business which had
occasioned a great deal of trouble.

%% We provide a command for single-image figure that handles the display.
\begin{figure}[htbp]
\centerline{\IncludeGraphics{height=5cm}{EdgarAllanPoe.jpg}} % instead of \includegraphics[...]{...} 
\caption{This is how the figure displays... this is Edgar Allan Poe.\label{fig:example}}
\end{figure}


\end{document}
